\documentclass[12pt]{article}

\usepackage[margin=1in, paperwidth=8.5in, paperheight=11in]{geometry}
\usepackage[table,xcdraw]{xcolor}
\usepackage{graphicx}
\usepackage{amsmath}
\usepackage{sidecap}
\usepackage{caption}
\usepackage{subcaption}
\usepackage[table,xcdraw]{xcolor}

\begin{document}
\begin{center}
Jeffrey Rodriguez 110733867\\AMS 326\\Test 3\\4/12/2018\\

\end{center}
\section*{Test 3-1}
This problem asks us to consider a company with initial value \$10M, and final value \$100B after 13 years of trading. We are given $\sigma = 0.1473\%$ and time, and must first find $\mu$. After this, we will plot the stock values at the end of each year with respect to year. Next, we look at the value of the GDP, with mean $\mu = 0.03385\%$ and $\sigma = 0.01414$. Finally, we look for the time year where the GDP exceeds \$20.20T.
\\We guess three values for $\mu$: 0.0027, 0.002725, 0.00275, and perform a quadratic interpolation with these points (similar to that in homework 2 and 3). After obtaining a polynomial and shifting it by -100B, we solve for the roots with a numpy command, and find $mu\approx0.0027122$. Next is the plot of NHA's stocks with respect to time.
\begin{figure}[h]
	\includegraphics*[scale = .7]{t3plot.png}
\end{figure}
\\On the next page is a plot of GDP values with respect to year again.
\newpage
\begin{figure}[h]
	\includegraphics*[scale = .7]{t3plot2.png}
\end{figure}
\end{document}