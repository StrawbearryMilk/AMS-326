\documentclass[12pt]{article}

\usepackage[margin=1in, paperwidth=8.5in, paperheight=11in]{geometry}
\usepackage[table,xcdraw]{xcolor}
\usepackage{graphicx}
\usepackage{amsmath}
\usepackage{sidecap}
\usepackage{caption}
\usepackage[table,xcdraw]{xcolor}

\begin{document}
\begin{center}
Jeffrey Rodriguez 110733867\\AMS 326 Test 1 \\2/08/2018\\

\end{center}

\section*{Ti-j}
\subsection*{Introduction}
We wish to approximate the first five roots for the derivative of the sinc function. This is equivalent to $$f(x) = \frac{1}{x}(\cos(x)-\sin(x)/x)$$. This was plotting on Mathematica using the commands: 
\\ \texttt{$f[x_] = 1/x (Cos[x] - Sinc[x])
\\\text{Plot}[f[x], {x, -6\pi, 6 \pi}]$}
The resulting graph is shown below:
\begin{figure}[h]
	\begin{center}
		\includegraphics*[scale=1]{dsincplot.png}
	\end{center}
\end{figure}
\subsection*{Method}
We see from the graph that the first root must be zero. Next, we see that by symmetry, for some root $x_0$, $-x_0$ must also be a root. Using the bisection method, we guess a point near the next root to be $x_0 = 4.49 \pm 0.005$. The next estimation is $x_0 = 7.7 \pm 0.05$.
\\We use the bisection method on these points, and there negative points to find the roots via Python.
\subsection*{Results}
As stated previously, we can tell from the graph that there must be a root as $x$ approaches 0. As can be seen by running the attached code, the other four roots found via Python are $x = \pm4.493408203125$ following 9 iterations, and $x = \pm 7.725244140625$ after 10 iterations.
\subsection*{Conclusion}
The code runs fast, and we see that these values are close to 0. It should be easy to transfer this to another computer, and simply making $\epsilon$ smaller will allow for closer approximations.
\end{document}